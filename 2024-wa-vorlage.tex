\documentclass[paper=a4,fontsize=12pt,ngerman]{scrartcl}

\usepackage[utf8]{inputenc}				
\usepackage[T1]{fontenc}
\usepackage{graphicx}
\usepackage[ngerman]{babel}
\usepackage{amsmath}
\usepackage[a4paper,left=25mm,right=35mm,top=25mm,bottom=30mm]{geometry}
\usepackage{parskip}

\usepackage{tcolorbox}
\tcbuselibrary{listingsutf8}
\usepackage{listings}

\usepackage{verbatim} % in the preamble
\usepackage{subcaption}  % in your preamble
	

\begin{document}

\pagenumbering{roman}
\pagestyle{plain}

% Einbinden der Titelseite
\begin{titlepage}

\linespread{1.5}

\includegraphics[width=\linewidth]{graphics/htw_logo}

\begin{center}
    \large  
    \hfill
    \vfill
    \Large{\bfseries{Eine historische Betrachtung des Quicksort-Algorithmus}}
    
    von \\
    Klaus Müller

    \vfill
		
    Ein wissenschaftlicher Bericht im Rahmen der Vorlesung\\
    \glqq Wissenschaftliches Arbeiten\grqq\\
    an der htw saar im Studiengang Informatik\\
	
    \vfill	
    \vfill
	
    Saarbrücken, den 17.07.2020
\end{center}
    
\end{titlepage}


% Hier ist der Abstract
\section*{Abstract}



Dieser Bericht zielt darauf ab, aktuell benutzte Algorithmen der Netzwerkstaukontrolle in Computernetzwerken zu vergleichen.
In dem heutigen Stand von Rechnernetzen werden verschiedene Protokolle zur Netzwerkstaukontrolle verwendet, 
Ich werde mich im Rahmen dieses Berichts allerdings auf das Transmission Control Protocoll beschränken, da dies das am meisten verbreitete ist.  \newline
Die Algorithmen der Staukontrolle, welche in diesen Vergleich einfließen sind : TCP Reno, TCP BBR, TCP Vegas, und ECN.
\newline
Mittels des Open-Source Netzwerksimulators \textit{NS-3} wird ein virtuelles Netzwerk zwischen zwei Endpunkten aufgestellt, über
welches ein Künstlicher Datenverkehr erzeugt wird.
Der genannte Datenverkehr wird anschließend aufgezeichnet und dient somit als Basis für den Vergleich der Staukontrolle durch 
die verschiedenen Algorithmen.
Die Auswertung, welche aus den Aufzeichungen stammt, ermöglicht dem Leser ein tieferes Verständnis über Staukontrolle in Computernetzwerken
zu erhalten. 




\newpage
\section*{Selbstständigkeitserklärung}
Ich versichere, dass ich die vorliegende Arbeit selbstständig verfasst und 
keine anderen als die angegebenen Quellen und Hilfsmittel benutzt habe.
Insbesondere habe ich alle KI-basierten Werkzeuge angegeben, die ich bei
der Erstellung, Übersetzung oder Überarbeitung des Textes verwendet habe.

Ich erkläre hiermit weiterhin, dass die vorgelegte Arbeit zuvor weder von mir 
noch von einer anderen Person an dieser oder einer anderen Hochschule 
eingereicht wurde.

Darüber hinaus ist mir bekannt, dass die Unrichtigkeit dieser Erklärung eine 
Benotung der Arbeit mit der Note \glqq nicht ausreichend\grqq \ zur Folge hat 
und einen Ausschluss von der Erbringung weiterer Prüfungsleistungen zur Folge 
haben kann.
\bigskip
 
Saarbrücken, den \today

\smallskip
Unterschrift  Micha\l{} Roziel




% Das Inhaltsverzeichnis
\clearpage
\tableofcontents 

\clearpage
\pagenumbering{arabic}

% Hier beginnt das erste Kapitel


\section{Einleitung}

Network congestion control (CC)  \footnote{\textit{Network congestion control} (Deutsch : \textit{ Netzwerkstaukontrolle }) } ist ein essentieller Bestandteil der 
meisten modernen Computernetzwerke. 
Wenn eine Netzwerkschnittstelle zu einem Zeitpunkt versucht eine zu große Menge an Datenpaketen aufzunehmen,
kommt es zu Stau von Datenverkehr und zu einem potentiellen Verlust von Datenpaketen.
Aufgrund diesem Vorkommen werden Algorithmen innerhalb von Netzwerkprotokollen verwendet, diese erkennen den Anstau von Datenverkehr im Netzwerk,
und helfen den Fluss von Datenpaketen zu steuern. Neben dem effizienten Durchfluss von Informationen ist zeitgleich auch die Fairness, 
was die Verteilung von Ressourcen einer Netzwerkschnittstelle an ihre Hosts angeht, wichtig. Auch dies wird von CC-Algorithmen gewährleistet.
\newline
Mit stets weiterentwickelten Rechnernetzen, und einem jährlich zunehmenden Datenverkehr, wie auch in der Deutschen Internetschnittstelle
\textit{DE-CIX}\cite{DE-CIX2025} in Frankfurt gewinnen diese Algorithmen an Bedeutung. 





\subsection{Grundlagen und Begriffe}

\subsubsection{Transmission Control Protocoll}

Das Transmission Control Protocoll (TCP)\footnote{\textit{Das Transmission Control Protocoll} (Deutsch : \textit{Übertragungssteuerungsprotokoll})} ist ein weit verbreitetes Netzwerkprotokoll. 
TCP lässt sich in zu der Schicht 4 (Transportschicht) in dem OSI-Modell einordnen, hierbei liegt es zwichen der Vermittlungs- und Kommunikationsschicht.
\newline Es wurde 1981 unter RFC 793 erstmals standarisiert. \cite{rfc793}   % QUOTE RFC 793
Die Aufgabes des TCP Protokolls ist es, zwischen zwei Hosts eine Verbindung aufzubauen, welche anschließend dazu genutzt wird,  
Nachrichten in zu verschicken und zu empfangen.

\begin{quote}
``transport-layer protocol [\dots] from an application’s perspective, it is as if 
the hosts running the processes were directly connected;'' \cite[241]{kr22}.      
\end{quote}


Dies Bedeutet, dass TCP ebenfalls eine gewisses Abstraktionsniveau
des Nachrichtenaustausches abnimmt. 
\newline
Ein klares Unterscheidungsmerkmal des TCP von dem ebenfalls bekannten User Datagramm Protocoll (UDP)
ist, dass TCP verbindungsorientiert arbeitet, während UDP als verbindungslos gilt.
Bei TCP wird zu Beginn ein \textit{Three-Way-Handshake}\footnote{\textit{Three-Way-Handshake} (Deutsch : \textit{Drei-Wege-Handschlag})} 
wie folgt durchgeführt : 
\newline
Der Sender schickt zunächst eine Nachricht mit einem Verbindungswunsch an den Empfänger und setzt das Flag 
\textit{Synchronize, \textbf{SYN}}.
\newline
Der Empfänger antwortet mit einer Bestätigung die erste Nachrichte erhalten zu haben, und schickt ebenfalls einen 
Verbindungswunsch. Es werden die Flags \textit{\textbf{SYN-ACK}} (Synchronise-Acknowledge) gesetzt.
\newline
Als letztes schickt der Sender eine Bestätigung, dass die Nachricht des Empfängers angekommen ist, dies geschieht wieder
mit dem Flag \textit{\textbf{ACK}}. Nun ist die Verbindung bereit, Daten in beide Richtungen zu übertragen.
Nach jedem gesendeten Datenpaket folgt eine \textit{\textbf{ACK}} Bestätigung.

\begin{figure}[ht]
    \centering
    \includegraphics[height = 7cm]{./graphics/3way.png}
    \caption{\textit{Three-Way-Handshake.}}  \cite[296]{kr22}
\end{figure}



\subsection{Algorithmen der Staukontrolle }

Algorithmen der Staukontrolle kommen zum Einsatz, wenn in einem Netzwerk ein potentieller Datenstau erkannt wird.
Grundsätzlich starten die Algorithmen in dem Arbeitsmodus \textit{slow start}.
Falls Datenstau erkannt wird, treten diese Algorithmen in den Arbeitsmodus \textit{congestion avoidance}
\footnote{\textit{congestion avoidance} (Deutsch : \textit{Vermeidung von Netzwerküberlastung}).}
 Hiermit wird mit verschiedenen Vorgehensweisen gegen die Netzwerküberlastung gesteuert. 
\newline
Es existieren unterschiedliche Typen von CC-Algorithmen, die jeweils ihre eigenen Vor- und Nachteile haben.
In diesem Bericht wird jeweils ein Algorithmus pro Typ behandelt.

\subsubsection{Typen von Algorithmen der Staukontrolle}


    \begin{itemize}
        \item Loss-Based Algorithmen
        
        
        \item Delay-Based Algorithmen
        
        \item Model-Based Algorithmen
    \end{itemize}


\subsubsection{TCP Reno}

\subsubsection{TCP BBR} 

\subsubsection{TCP Vegas}

\subsubsection{ECN}




\subsection{Definitionen}

\subsubsection{Queuing Delay}
Queuing Delay beschreibt das Warten einer \textit{Queue}
auf das Versenden durch das gegebene Netzwerk.


\section{Versuchsaufbau}

Mit einem Test innerhalb des Command Line Simulators \textit{NS-3} lässt sich ein belieiger 
CC-Algorithmus ausführen.


\textbf{Netzwerktopologie}

Zunächst einmal definieren wir die Netzwerktopologie auf welcher die CC-Algorithmen
laufen werden.
\small
\begin{verbatim}
            1000 Mbps           10 Mbps           1000 Mbps
  Sender -------------- R1 -------------- R2 -------------- Receiver
            5 ms               10 ms              5 ms
\end{verbatim}

\begin{figure}[htp]


    \centering

    % Erster Algo : BBR
    \subcaptionbox{Congestion Window des BBR Algorithmus.
    \label{fig:bbr}}[0.43\textwidth]{
        \includegraphics[width=\linewidth]{graphics/bbrCW.pdf}
    }
    \hfill
    % Zweiter Algo : Reno (NewRen)
    \subcaptionbox{Congestion Window des NewReno Algorithmus.
    \label{fig:newreno}}[0.43\textwidth]{
        \includegraphics[width=\linewidth]{graphics/newRenoCW.pdf}
    }

    % bisschen frei lassen
    \par\vspace{0.8em} 

        \subcaptionbox{Congestion Window des Cubic Algorithmus.
    \label{fig:cubic}}[0.43\textwidth]{
        \includegraphics[width=\linewidth]{graphics/cubicCW.pdf}
    }

    \caption{Vergleich der Congestion Window-Verläufe zwischen BBR und NewReno.}
    \label{fig:cwnd-comparison}
\end{figure}

\subsection{Analyse der Ergebnisse}


\section{Ergebnisse und Diskussion}


\section{Fazit und Schlussfolgerungen}


\subsection{Offene Fragen}


\subsection{Diskussion}


% Hier beginnt das Literaturverzeichnis
\clearpage
\renewcommand\refname{Literaturverzeichnis}
\bibliographystyle{alpha}
\bibliography{literatur}
\addcontentsline{toc}{section}{Literaturverzeichnis}


% Hier beginnt der Anhang
\clearpage
\appendix
\part*{Anhang}
\addcontentsline{toc}{section}{Anhang}

\section{Datenmaterial}


\section{Web-Standards}
O

\end{document}

